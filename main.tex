\documentclass{article}
\usepackage[utf8]{inputenc}

\title{Home assignment 2}
\author{Numerical Optimization and its Applications - Spring 2019\\Gil Ben Shalom, 301908877\\Tom Yaacov, 305578239}
\date{\today}

\usepackage[shortlabels]{enumitem}
\usepackage{pythontex}
\usepackage[most]{tcolorbox}
\usepackage{amsmath,amsthm,amssymb}

\newcommand{\importandtypeset}[1]{
  %\pyc{print(12);add_to_path('py_files'); from #1 import *; pytex.add_dependencies(os.path.join('py_files', '#1.py'))}%
  \inputpygments{python}{py_files/#1.py}%
  \pyc{from py_files import #1}
}

\newcommand{\saveandshowplot}[1]{
  \begin{center}
  \includegraphics[width=0.85\textwidth]{#1}
  \end{center}
}

\begin{document}

\maketitle

\section{The efficiency of different iterative methods for solving a linear system}
\begin{enumerate}[(a)] 
\item 
Following are the implementation for the four methods:\\
\textbf{Jacobi:}
\begin{tcolorbox}
\importandtypeset{part_1_jacobi}
\end{tcolorbox}
\textbf{Gauss-Seidel:}
\begin{tcolorbox}
\importandtypeset{part_1_gauss_seidel}
\end{tcolorbox}
\item Following are the system and parameters definition, methods calls, residual vector norm and convergence factor plotting:
\begin{tcolorbox}[%
    enhanced, 
    breakable,
    frame hidden,
    overlay broken = {
        \draw[line width=0.5mm, black, rounded corners]
        (frame.north west) rectangle (frame.south east);},
    ]{}
\importandtypeset{part_1_b}
\end{tcolorbox}
\end{enumerate}
\section{Convergence properties}
\begin{enumerate}[(a)] 
\item t
\item t
\item t
\end{enumerate}
\section{GMRES(1) method}
\begin{enumerate}[(a)] 
\item t
\item t
\item t
\item t
\item t
\end{enumerate}
\section{Convexity}
\begin{enumerate}[(a)] 
\item t
\item t
\item t
\end{enumerate}
\section{Non Linear Optimization}
\begin{enumerate}[(a)] 
\item t
\item t
\item t
\end{enumerate}
\end{document}
